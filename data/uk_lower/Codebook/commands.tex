%----------------------------------------------------------------------%
% command to create a new variable entry
%----------------------------------------------------------------------%

\newcommand{\entry}[2]{\variable{#1} \hspace{10pt} #2 \vspace{5pt}}

%\newcommand{\entry}[2]{\variable{#1} \hspace{10pt} #2}

%----------------------------------------------------------------------%
% commands to create common variable entries
%----------------------------------------------------------------------%

\newcommand{\observationnumber}[1]{\entry{observation\_number}{The number of the observation. Indicates the default sorting order. Sorted by #1.}}

\newcommand{\observationpath}[2]{\entry{observation\_path}{A path that uniquely identifies each observation in the dataset in the format \code{#1}. Identical in this dataset to \code{#2}.}}

%----------------------------------------------------------------------%
% paths
%----------------------------------------------------------------------%

\newcommand{\pathentry}[3]{\entry{#1}{A path that uniquely identifies each #2 in the format \code{#3}.}}

%----------------------------------------------------------------------%
% numbers
%----------------------------------------------------------------------%

\newcommand{\parliamentnumber}{\entry{parliament\_number}{The number of the parliament.}}

\newcommand{\sessionnumber}{\entry{session\_number}{The number of the session within a parliament. This variable does not uniquely identify sessions because session numbers repeat across parliaments.}}

\newcommand{\chambernumber}{\entry{chamber\_number}{A number assigned to each chamber. The House of Commons is coded \code{1} and the Senate is coded \code{2}.}}

\newcommand{\chambername}{\entry{chamber\_name}{The name of the chamber.}}

\newcommand{\sittingnumber}{\entry{sitting\_number}{The number of the sitting within a session. This variable does not uniquely identify sittings because sitting numbers repeat across chambers and sessions.}}

\newcommand{\constituencynumber}{\entry{constituency\_number}{A number assigned to each chamber. Assigned with constituencies sorted by name.}}

\newcommand{\speechnumber}{\entry{speech\_number}{The number of the speech within a sitting. Note that this variable does not uniquely identify speeches because speech numbers repeat across sittings.}}

\newcommand{\paragraphnumber}{\entry{paragraph\_number}{The number of the paragraph of the speech. Note that this variable does not uniquely identify paragraphs because paragraph numbers repeat across speeches.}}

\newcommand{\committeenumber}{\entry{committee\_number}{A number assigned to each committee within a session. Assigned with committees sorted by name.}}

\newcommand{\committeesittingnumber}{\entry{sitting\_number}{The number of the committee sitting within a session. This variable does not uniquely identify sittings because sitting numbers repeat across committees and across sessions.}}

\newcommand{\membernumber}{\entry{member\_number}{A number assigned to each member. Assigned with members sorted by last name, then by first name, then by constituency. Not comparable across groups.}}

\newcommand{\questionnumber}{\entry{question\_number}{The number of the question. Note that this variable does not uniquely identify questions because question numbers repeat across sittings.}}

\newcommand{\eventnumber}{\entry{event\_number}{The number of the event. Note that this variable does not uniquely identify events because event numbers repeat across items.}}

\newcommand{\motionclassnumber}{\entry{motions\_class\_number}{A number assigned to each class of motion. Private members' motions are coded \code{1}, motions for the production of papers are coded \code{2}, government motions are coded \code{3}, and ways and means motions are coded \code{4}.}}

\newcommand{\motionnumber}{\entry{motion\_number}{The number of the motion. Note that this variable does not uniquely identify motions because motion numbers repeat across classes of motions and sittings.}}

\newcommand{\billnumber}{\entry{bill\_number}{The number of the bill. Note that this variable does not uniquely identify bills because bill numbers repeat across chambers and sessions.}}

\newcommand{\divisionnumber}{\entry{division\_number}{The number of the division. This variable does not uniquely identify divisions because division numbers repeat across sessions.}}

\newcommand{\votenumber}{\entry{vote\_number}{The number of the individual vote. Assigned with members sorted by last name, then by first name, then by constituency. This variable does not uniquely identify individual votes because vote numbers repeat across divisions.}}

\newcommand{\ministrynumber}{\entry{ministry\_number}{The number of the ministry.}}

\newcommand{\ministernumber}{\entry{minister\_number}{A number assigned to each minister in a ministry. Assigned with ministers sorted by last name, then by first name, then by constituency. This variable does not uniquely identify ministers because minister numbers repeat across ministries.}}

\newcommand{\portfolionumber}{\entry{portfolio\_number}{A number assigned to each portfolio of a minister. Assigned with portfolios sorted by name. This variable does not uniquely identify portfolios because portfolio numbers repeat across ministers.}}

\newcommand{\electionnumber}{\entry{election\_number}{A number assigned to each election. Assigned with elections sorted by date.}}

\newcommand{\racenumber}{\entry{race\_number}{A number assigned to each race in an election. Assigned with races sorted by constituency name. This variable does not uniquely identify races because race numbers repeat across elections.}}

\newcommand{\candidatenumber}{\entry{candidate\_number}{A number assigned to each candidate in a race. Assigned with candidates sorted by last name, then by first name. This variable does not uniquely identify candidates because candidate numbers repeat across races.}}

\newcommand{\itemnumber}{\entry{item\_number}{A number assigned to each item. Note that this variable does not uniquely identify items because item numbers repeat across parliaments.}}

\newcommand{\versionnumber}{\entry{version\_number}{The version of the bill. Note that this variable does not uniquely identify versions because versions numbers repeat across bills.}}

%----------------------------------------------------------------------%
% ID variables
%----------------------------------------------------------------------%

% bill

\newcommand{\billid}{\entry{bill\_ID}{A path that uniquely identifies each bill in the format \code{/parliament-41/session-2/chamber-1/bill-1}. See \code{canada\_bills.csv}.}}

% motion

\newcommand{\motionid}{\entry{motion\_ID}{A path that uniquely identifies each motion in the format \code{/parliament-41/session-2/chamber-1/motion-class-1/motion-1}. See \code{canada\_motions.csv}.}}

% question

\newcommand{\questionid}{\entry{question\_ID}{A path that uniquely identifies each question in the format \code{/parliament-41/session-2/chamber-1/question-1}. See \code{canada\_questions.csv}.}}

% member

\newcommand{\membername}[1]{\entry{member\_name}{The full name of the member#1. This variable does not uniquely identify members because members can have the same name.}}

\newcommand{\memberid}{\entry{member\_ID}{A path segment that uniquely identifies each member of the House of Commons in the format \code{/chamber-1/member-1}. See \code{canada\_members.csv}.}}

% constituency

\newcommand{\constituencyname}[1]{\entry{constituency\_name}{The name of the constituency of the member#1.}}

\newcommand{\constituencyid}{\entry{constituency\_ID}{A path segment that uniquely identifies each constituency in the format \code{/chamber-1/constituency-1}. See \code{canada\_constituencies.csv}.}}

% committee

\newcommand{\committeename}{\entry{committee\_name}{The name of the committee.}}

\newcommand{\committeeacronym}{\entry{committee\_acronym}{The acronym for the committee.}}

\newcommand{\committeeid}{\entry{committee\_ID}{A path segment that uniquely identifies each committee in the format \code{/chamber-1/committee-1}. See \code{canada\_committees.csv}.}}

%----------------------------------------------------------------------%
% command to create a heading
%----------------------------------------------------------------------%

%\newcommand{\heading}[1]{\newpage \vspace{40pt} \hrule \vspace{0pt} \code{#1} \vspace{10pt} \hrule \vspace{10pt}}

\newcommand{\heading}[3]{

\newpage

\phantomsection

{\color{mypurple} \huge \textbf{#1}}

{\code{#2} \hspace{10pt} \code{#3}}

\addcontentsline{toc}{section}{#1}

}

\newcommand{\simpleheading}[1]{

\newpage

\phantomsection

{\color{mypurple} \huge \textbf{#1}}

\addcontentsline{toc}{section}{#1}

}

\newcommand{\subheading}[1]{{\color{mypurple}\large\textbf{#1}}}

\newcommand{\newheading}[6]{
\newpage
\vspace{0pt} \hrule \vspace{5pt}
\textbf{\thesection.}\hspace{10pt}\textbf{#1} 
\vspace{15pt} \hrule \vspace{5pt}
File name: \code{#2} \\
File size: \code{#3} \\ 
Number of observations: \code{#4} \\ 
Number of variables: \code{#5}
\vspace{15pt} \hrule \vspace{5pt}
}

%----------------------------------------------------------------------%
% command to format variable values
%----------------------------------------------------------------------%

\newcommand{\+}[1]{\item \code{#1}}







